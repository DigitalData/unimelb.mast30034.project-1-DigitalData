% Setup - do not change
\documentclass[11pt]{article}
\usepackage[top=0.9in, left=0.9in, bottom=0.9in, right=0.9in]{geometry} 
\usepackage{parskip}
\usepackage[english]{babel}
\usepackage[utf8]{inputenc}
\usepackage{amsmath,amsthm,amssymb,graphicx,pdfpages,lipsum,hyperref}
\usepackage[none]{hyphenat}
\usepackage{csquotes}

\setlength\parindent{0pt}
%%%%%%%%%%%%%%%%%%%%%%%%%%%%%%%%%%%%%%%%%%%%%%%%%%%%%%%%%%%%%%%%%%%
% add other packages here if required

%% Bibliography are specified in this file. You can also choose inline bib style if you want to. But make sure your citation style is consistent (and proper)
% For more details on citation: https://library.unimelb.edu.au/recite
\usepackage[sorting = none]{biblatex}
\addbibresource{references.bib}

%%%%%%%%%%%%%%%%%%%%%%%%%%%%%%%%%%%%%%%%%%%%%%%%%%%%%%%%%%%%%%%%%%% the '%' symbol denotes comments

% Begin document creation
% DELETE THE \lipsum PLACEHOLDERS WHEN YOU BEGIN
\title{\textbf{Insert Title} \\ Insert Subtitle}
\author{
Xavier Travers \\
Student ID: 1178369 \\
%% Replace the link with your github repo
% 1. Remember to escape underscore in the link.
% 2. Remember to include the commit you want to submit in the link
\href{https://github.com/MAST30034-Applied-Data-Science/mast30034\_p1\_template/tree/fd9f1dd17fdbcb5b119b70c93a22da8210d44fd7}{Github repo with commit}
}

\begin{document}
\maketitle

\section{Introduction}
% Link to a 30 min tutorial if you require revision: https://www.overleaf.com/learn/latex/Learn_LaTeX_in_30_minutes

Throughout this research project, I will be investigating the impact that the COVID-19 pandemic has had on the use of paid transport services.


% \LaTeX{} Have many caveats, you should search stack overflow for latex tips whenever you feel something looks bad, for instance:
% When `` quoting '', should be used instead of ". For example, ``test'' vs "test".

% % use \textbf{} for bold text and \textit{} for italic. 
% % \texttt{} creates code blocks akin to `code ticks` in markdown
% \textbf{Please refer to the spec, the word count and page count is strict.} Feel free to change the section headings (and we recommend you do).

% Always remember to cite materials that does not belong to you. For instance, you should cite the sensor datasets \cite{2022sensorreading, 2022sensorlocation}.
% % Example here used biblatex to manage citations: https://www.overleaf.com/learn/latex/Bibliography_management_with_biblatex , You are free to choose your own way for managing references if biblatex seems too hard.

% \lipsum[7]

% You can have \section{}, \subsection{}, and \subsubsection{}
\section{Preprocessing, Analysis, and Geospatial Visualisation}
\begin{enumerate} 
    \item Example for enumerated points
    % use \item to create more points
\end{enumerate}

\begin{itemize} 
    \item Example for dot points
    \item[*] You can change dot points to any symbols by putting [SYMBOL].
    \item[$\times$] Here's a fun example.
\end{itemize} 
\lipsum[4-5]
Example code for figures:
% the [h] ensures your figure is inline at the location and not displayed on some other page
\begin{figure}[h]
    % change the scale multiplier to make the figures smaller or larger
    \includegraphics[width=0.35\textwidth]{example-image-a}
    % this ensures your figures are centered where possible
    \centering
    \caption{Some caption} % refer to this image as (Figure 1)
\end{figure}
\lipsum[1-2]

\section{Modelling}
Example of a maths equation:
\begin{equation}
    Y = X\beta + \epsilon
\end{equation}

Example of an aligned equation (\& denotes the symbol to align):
\begin{align*}
    E[\mathbf{y}] &= X\beta + E[0] \\
                  &= X\beta
\end{align*}

Example of an in-line equation $\epsilon \sim N(0, 1)$ \\

\section{Recommendations}
\lipsum[10]

\section{Conclusion}
\lipsum[14-15]


\clearpage

% BEGIN REFERENCES SECTION
\printbibliography

\end{document}